\documentclass[french]{article}
\usepackage[utf8]{inputenc}
\usepackage[T1]{fontenc}
\usepackage[a4paper]{geometry}
\usepackage{babel}
\begin{document}


\title{\textbf{Guerre et Sel}}
\date{\today}
\author{Fumelgo}

\maketitle
\newpage

\tableofcontents

\newpage


\section{Préambule}

Ce jeu de rôle a pour vocation de pouvoir être adapté à n'importe quelle guerre (hormis les guerres modernes). Ici, le thème choisi est l'antiquité européenne. Ce choix est motivé par la facilité d'accès aux informations sur ce sujet. Le but est de rester le plus Jdr possible, tout en fournissant une façon de donner vie à des rixes comportant plusieurs dizaines de milliers d'hommes !
Ainsi, chaque joueur aura une liberté totale dans ses choix comme dans tout jdr. Une partie application avait été prévue, mais a été abandonnée.


\section{Mécaniques}

\subsection{Fiche Joueur}

Le joueur possède une fiche. Sur celle-ci il pourra : dessiner son joueur, y noter ses hommes, ses ressources, sa richesse, ses caractéristiques. Les divisions sont déjà traitées, la richesse sera en milliers de Yuan, et il reste à expliquer les caractéristiques.


\begin{itemize}
	\item \textbf{Grade} : Votre grade dans l'armée. Celui-ci défini directement le nombre d'hommes que vous êtes censé avoir sous vos ordres (niveau)
	\item \textbf{Renommée} : Votre réputation dans votre nation et au-delà des frontières. Définira la qualité des hommes qui se présenteront à vos recrutements, et vous permettra de gagner en grade (expérience)
	\item \textbf{Charisme} : Une qualité première d'un général est son charisme. Même si certains sont de brillants tacticiens et sont arrivés à se poste sans charisme. Inspire vos hommes au combat. (attaque, force)
	\item \textbf{Puissance} : Certains généraux sont très puissants et valent une division à eux seuls. Vous pouvez faire la différence au front.
	\item \textbf{Intelligence} : Un esprit mieux aiguisé qu'une épée peut sauver de nombreuses vies. Des détails dans les mouvements des ennemis vous arrivent plus facilement, vous élaborez des plans complexes.
	\item \textbf{Réseau} : Certains lieutenants se démarquent par le fait de toujours avoir un coup d'avance. Disposez des meilleurs espion du royaume et ne soyez jamais surpris lors de vos campagnes.
\end{itemize}


\subsection{Attaque}

Les dégâts de l'attaque seront influencées par :
\begin{itemize}
	\item Les caractéristiques du lieutenant (le joueur)
	\item Le lancer de dé du lieutenant
	\item Le côté attaqué (avant, côté, arrière)
	\item Le nombre de guerriers de chaque côté
	\item Leur valeur
	\item Leur souffle
	\item Leur moral
	\item Le milieu (bonus pour une attaque en descente)
	\item Le type de milieu (boisé, fortifié, etc), et si les hommes sont formés pour combattre dedans
	\item Le ravitaillement
	\item Type d'attaque (charge ou approche lente ?)
\end{itemize}

Une attaque critique ou abaissant trop le moral peut mener une division en déroute.
Chaque attaque diminue la vigueur des participants en fonction de son type.


\subsection{Espionnage}

Pour mieux pouvoir combattre, lorsque vous n'êtes pas en campagne, vous pouvez espionner, écouter les nouvelles. Certaines armées sont célèbres pour leur cavalerie, mais d'autres sont plus mystiques... A vous de choisir si oui ou non vous êtes prêts à dépenser une partie de votre argent dans le recueil d'informations.
Attention, l'espionnage se fait par jet, et ainsi donc vous risquez d'obtenir des informations peu fiables... (Sauf coup critique, évidemment !)
Un échec critique et vous perdez votre agent (malus en réseau...)


\subsection{Campagne}

Le jeu aura malheureusement un schéma assez simple. Les joueurs commencent assez bas dans les échelons. Ils aspirent chacun à devenir des généraux et changer l'histoire de la Chine (qui sait, un autre état au pouvoir, une chine qui conquit la terre entière ?)
Ils sont chacun à la tête d'une division (dont la taille dépend du grade, et le type est choisit en fonction des stats (un peu comme la race)).
Le cycle est simple. Recruter, espionner. Partir en guerre, enchaîner des batailles. Enterrer les morts. Célébrer une victoire, pleurer la défaite. Renforcer les troupes. Gagner en grade, etc. En plus, il faudra se déplacer de champ de bataille en champ de bataille, en fonction du temps imparti...

Le rôle principal du MJ sera de créer les campagnes, et les éventuels lieux de batailles, les armées adverses. Ensuite, il servira d'arbitre pendant les batailles.



\section{Lancés de dés}

\section{Création du personnage}

Voici un exemple de grades, avec le niveau de renommée nécessaire et le nombre d'hommes sous vos ordres, à titre indicatif. En fonction du thème du JDR, le nom des grades va changer.

\begin{tabular}{| c | c | c |}
\hline
\textbf{Grade} & \textbf{Renommée} & \textbf{Hommes}\\
\hline
Vice-Capitaine & 0 & 100\\
\hline
Capitaine & 100 & 500\\
\hline
Lieutenant-Capitaine & 300 & 1500\\
\hline
Vice-Lieutenant & 600 & 3000\\
\hline
Lieutenant & 1000 & 8000\\
\hline
Lieutenant-Général & 1500 & 10.000\\
\hline
Général & 2100 & 3*10.000\\
\hline
Grand Général & 2800 & 7*10.000\\
\hline
\end{tabular}

Le joueur aura 1D6+7 de charisme de base.\\
Le joueur aura 1D6+7 de puissance de base.\\
Le joueur aura 1D6+7 d'intelligence de base.\\
Le joueur aura 1D6+7 de réseau de base.\\

Sa richesse initiale sera liée à son intelligence et son réseau. Il aura (1D4 + 2)*1000 + (intelligence * réseau * 100).

Voici le tableau des types d'unités que le joueur pourra prendre sous son commandement :


\begin{tabular}{| c | c | c | c |}
\hline
\textbf{Type} & \textbf{Charisme} & \textbf{Puissance} & \textbf{Intelligence}\\
\hline
Infanterie Lourde & 10+ & 10+ & \\
\hline
Archers & & & 11+\\
\hline
Cavalerie & & 10+ & 10+\\
\hline
Lanciers & & 11+ & \\
\hline
Saboteurs & 10+ & & 10+\\
\hline
Infanterie & & &\\
\hline
\end{tabular}

Le joueur pourra plus tard avoir plusieurs divisions sous son ordre, évidemment (lorsqu'il aura atteint le grade nécessaire).

\section{Recrutement}

Vous aurez naturellement des soldats attribués par l'état avant chaque campagne. Éventuellement, vous recevrez des unités de divisions qui ont perdu leur commandant à la bataille précédente.
Vous pourrez faire des campagnes de recrutement, et grâce à votre charisme (lancé de 1D4 * centaines de points de renommée + charisme ) vous définirez la valeur des hommes qui arrivent. La nouvelle valeur de l'unité est une moyenne pondérée.
En payant, on peut améliorer la qualité des hommes attirés (publicité). 1000 yuans vaut un +1.

\section{Espionnage}

Jet d'espionnage 1D20, peut être paré par du contre-espionnage.

\section{Attaque}


\section{Action diverses}






\end{document}
